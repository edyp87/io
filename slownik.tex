%%%%%%%%%%%%%%%%%%%%%%%%%%%%%%%%%%%%%%%%%%%%%%%%%%%%%%%%%%%%%%%%%%%%%%%%%%%%%%%%%%%%%%%%%%
%%%%%%%%%%%%%%%                   PREAMBUŁA / USTAWIENIA               %%%%%%%%%%%%%%%%%%%

\documentclass [11pt, a4paper, leqno]	{article}	% czcionka 11pt, a4, numery formuł do lewej, artykuł
\usepackage [polish]	{babel} 					% ustawiamy jezyk polski dokumentu wynikowego
\usepackage 			{polski}					% zapewniamy sobie poprawne lamanie polski wyrazow
\usepackage [utf8]		{inputenc}					% kodowanie pliku zrodlowego
\usepackage [T1]		{fontenc}					% kodowanie fontow
\usepackage 			{indentfirst}				% wciecie do kazdego akapitu
\usepackage				{setspace}					% interlinia: 1.15
\setstretch {1.15}
\frenchspacing										% przestrzen po zakonczeniu zdania
\renewcommand {\thesection} 		{\arabic{section}.}       	% chcemy kropki na koncu numeracji sekcji
\renewcommand {\thesubsection} 		{\arabic{section}. \arabic{subsection}.}
\renewcommand {\thesubsubsection} 	{\arabic{section}. \arabic{subsection}. \arabic{subsubsection}.}
\textheight 620pt

%%%%%%%%%%%%%%%%%%%%%%%%%%%%%%%%%%%%%%%%%%%%%%%%%%%%%%%%%%%%%%%%%%%%%%%%%%%%%%%%%%%%%%%%%

\begin{document}

% STRONA TYTUŁOWA

\begin{center}
	\thispagestyle{empty} 							% brak numeracji, naglowka i stopki
	{\large Studencka Pracownia Inżynierii Oprogramowania} 		\\ [0.5cm]
	{\large Instytut Informatyki Uniwersytetu Wrocławskiego} 	\\ [6.0cm]

	{\large Mateusz Kawa, Maciej Przybecki, Marek Jenda} 		\\ [1.5cm]

	{\huge Dokumentacja witryny internetowej sklepu} 			\\ [0.5cm]
	{\huge NUMIZMATYKA} 										\\ [1.5cm]

	{\large Słownik} 										 	\\ [0.5cm]

	\vfill
	
	{\large Wrocław, \today}									\\ [0.5cm]
	{\large Wersja 0.2}
\end{center}

\newpage

% TABELKA WERSJI

\textit{Tabela 0.} Historia zmian dokonanych w dokumencie

\begin{center}
	\begin{tabular}{| l | l | l | l |}
		\hline
		\multicolumn{1}{|c|}{Data} & 
		\multicolumn{1}{|c|}{Numer wersji} &  
		\multicolumn{1}{|c|}{Opis} &
		\multicolumn{1}{|c|}{Autor} \\ \hline \hline
		2013-10-21 & 0.1 & Utworzenie dokumentu & Marek Jenda \\ \hline
		2013-11-02 & 0.2 & Korekta & Mateusz Kawa \\ \hline
	\end{tabular}
\end{center}
\newpage

\section{Słownik}
\begin{description}

\item[baza danych] --- system przechowujący dane w sposób zorganizowany. Organizacja bazy danych jest określana za pomocą zbioru ustalonych reguł. 

\item[hosting] --- udostępnianie zasobów serwerowych; umożliwia firmom, przedsiębiorstwom i klientom indywidualnym udostępnianie treści w internecie.

\item[HTML (\textnormal{ang.} \textit{HyperText Markup Language})] ---  system znaczników służących do tworzenia stron internetowych.

\item[implementacja] --- wdrożenie abstrakcyjnego opisu systemu lub programu.

\item[interfejs graficzny] --- rodzaj interfejsu użytkownika, który umożliwia interakcję użytkownika z programem poprzez zestaw graficznych elementów.

\item[język programowania] --- język formalny powstały w celu określenia instrukcji wykonywanych przez maszynę (komputer). 
Język ten może służyć jako sposób komunikowania komputerowi instrukcji, które ma wykonać. W innym znaczeniu język programowania jest wykorzystywany jako sposób precyzyjnej ekspresji algorytmów.

\item[kod źródłowy] --- ciąg instrukcji komputera zapisanych w postaci czytelnego dla człowieka języka programowania. 

\item[użytkownik] --- osoba lub osoby korzystające bezpośrednio z produktu.

\item[witryna] --- zbiór stron internetowych, zwykle komercyjnych, eksponujących coś; np. witryna sklepu internetowego. W niniejszej dokumentacji hasło to jest stosowane zamiennie wraz z określaniem w ten sposób przedmiotu tej dokumentacji (tj. witryny internetowej sklepu poświęconego tematyce numizmatycznej).


\end{description}

\end{document}