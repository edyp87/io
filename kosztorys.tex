%%%%%%%%%%%%%%%%%%%%%%%%%%%%%%%%%%%%%%%%%%%%%%%%%%%%%%%%%%%%%%%%%%%%%%%%%%%%%%%%%%%%%%%%%%
%%%%%%%%%%%%%%%                   PREAMBUŁA / USTAWIENIA               %%%%%%%%%%%%%%%%%%%

\documentclass 	[11pt, a4paper, leqno]	{article}					% czcionka 11pt, a4, numery formuł do lewej, artykuł
\usepackage 	[polish]		{babel} 					% ustawiamy jezyk polski dokumentu wynikowego
\usepackage 				{polski}					% zapewniamy sobie poprawne lamanie polski wyrazow
\usepackage 	[utf8]			{inputenc}					% kodowanie pliku zrodlowego
\usepackage 	[T1]			{fontenc}					% kodowanie fontow
\usepackage 				{indentfirst}					% wciecie do kazdego akapitu
\usepackage				{setspace}					% interlinia: 1.15
\usepackage{amsmath}
\setstretch 	{1.15}
\frenchspacing										% przestrzen po zakonczeniu zdania
\renewcommand 				{\thesection} 		{\arabic{section}.}     % chcemy kropki na koncu numeracji sekcji
\renewcommand 				{\thesubsection} 	{\arabic{section}. \arabic{subsection}.}
\renewcommand 				{\thesubsubsection} 	{\arabic{section}. \arabic{subsection}.\arabic{subsubsection}.}
\textheight 	620pt
\makeatletter
\renewcommand\@biblabel[1]{#1}
\makeatother

\usepackage{ifthen}
\newcommand{\newoddside}{								% tworzymy polecenie przenoszace ciag dalszy tekstu na kolejna nieparzysta strone
	\ifthenelse{ \NOT \isodd{\thepage} } {
	    \newpage
	} {
	    \newpage
	    \newpage
	}
}

%%%%%%%%%%%%%%%%%%%%%%%%%%%%%%%%%%%%%%%%%%%%%%%%%%%%%%%%%%%%%%%%%%%%%%%%%%%%%%%%%%%%%%%%%

\begin{document}

% STRONA TYTUŁOWA

\begin{center}
	\thispagestyle{empty} 							% brak numeracji, naglowka i stopki
	{\large Studencka Pracownia Inżynierii Oprogramowania} 		\\ [0.5cm]
	{\large Instytut Informatyki Uniwersytetu Wrocławskiego} 	\\ [6.0cm]

	{\large Mateusz Kawa, Maciej Przybecki, Marek Jenda} 		\\ [1.5cm]

	{\huge Dokumentacja witryny internetowej sklepu} 			\\ [0.5cm]
	{\huge NUMIZMATYKA} 										\\ [1.5cm]

	{\large Kosztorys} 										\\ [0.5cm]

	\vfill
	
	{\large Wrocław, \today}									\\ [0.5cm]
	{\large Wersja 0.3}
\end{center}

\newpage

% TABELKA WERSJI

\textit{Tabela 0.} Historia zmian dokonanych w dokumencie

\begin{center}
	\begin{tabular}{| l | l | l | l |}
		\hline
		\multicolumn{1}{|c|}{Data} & 
		\multicolumn{1}{|c|}{Numer wersji} &  
		\multicolumn{1}{|c|}{Opis} &
		\multicolumn{1}{|c|}{Autor} \\ \hline \hline
		2013-11-26 & 0.1 & Utworzenie dokumentu & Maciej Przybecki \\ \hline
		2013-12-03 & 0.2 & Usunięcie błędów technicznych & Maciej Przybecki \\ \hline
		2013-12-10 & 0.3 & Naniesienie poprawek & Mateusz Kawa \\ \hline
	\end{tabular}
\end{center}
% \newpage
\medskip

% SPIS TRESCI

\tableofcontents

\newpage

\section{Wprowadzenie}

\subsection{Cel dokumentu}
\noindent
Niniejszy dokument zawiera kosztorys realizacji projektu witryny internetowej NUMIZMATYKA. \\

\section{Kosztorys}
\noindent
Realizacja projektu zgodnie z dobrą praktyką inżynierii oprogramowania została podzielona na kilka etapów. Ze względu na rozmiar przedsięwzięcia oraz dobrą dokumentację pokrewnych projektów przyjęty został model kaskadowy. Prace postępują w sposób sekwencyjny, a każdy etap ma ustalony czas rozpoczęcia i czas zakończenia, których należy przestrzegać. \\

Projekt został podzielony na następujące etapy:
\begin{itemize}
\item analiza,
\item projektowanie,
\item implementacja,
\item testowanie,
\item ocena i przekazanie.
\end{itemize}


\subsection{Etap analizy}
\noindent
Na tym etapie są wymagane badania rynku oraz zakup odpowiednich narzędzi, z pomocą, których będzie projektowany system. \\ 

\textit{Tabela 1.} Koszty etapu analizy
\begin{center}
	\begin{tabular}{| l | l |}
		\hline
		\multicolumn{1}{|c|}{Przedsięwzięcie} & 
		\multicolumn{1}{|c|}{Przewidywany koszt} \\ \hline \hline
		Ankieta elektroniczna & 5 tys. zł \\ \hline
		Zakup niezbędnych programów & 10 tys. zł \\ \hline
	\end{tabular}
\end{center}

\subsection{Etap projektowania}
\noindent
Podczas tego etapu wymagane jest zatrudnienie 3 osób odpowiedzialnych za zaprojektowanie całego systemu. \\
\begin{equation}3 \text{[osoby]} \times  8500 \text{[zł/miesiąc]} \times 1 \text{[miesiąc]} = 25\\\ 500 \text{[zł]} \end{equation} \\
Wykonany jest także zakup odpowiednich narzędzi, za pomocą których będzie implementowany system.  \\
Zatrudnione będą także dwie osoby na stanowisko grafika komputerowego. Odpowiedzialne one będą za zaprojektowanie interfejsu użytkownika. \\
\begin{equation}2 \text{[osoby]} \times 2000 \text{[zł/miesiąc]} \times 1 \text{[miesiąc]} = 4000 \text{[zł]} \end{equation} \\
\textit{Tabela 2.} Koszty etapu projektowania

\begin{center}
	\begin{tabular}{| l | l |}
		\hline
		\multicolumn{1}{|c|}{Przedsięwzięcie} & 
		\multicolumn{1}{|c|}{Przewidywany koszt} \\ \hline \hline
		Zakup serwera & 50 tys. zł \\ \hline
		Honoraria dla projektantów & 25,5 tys. zł \\ \hline
		Zakup niezbędnych narzędzi & 20 tys. zł \\ \hline
		Honoraria dla grafików komputerowych & 4 tys. zł \\ \hline
	\end{tabular}
\end{center}

\subsection{Etap implementacji}
\noindent
Podczas tego etapu każdy moduł programu będzie implementowany przez oddzielną grupę programistów. \\
Przewidywane są 3 grupy programistów składające się z trzech osób, stawka za miesiąc na osobę wynosi 4000 zł. \\
\begin{equation}3 \text{[grupy]} \times 3 \text{[osoby]} \times 4000 \text{[zł/miesiąc]} \times  4 \text{[miesiąc]} = 144\\\ 000 \text{[zł]} \end{equation} \\
Zatrudnione będą także dwie osoby na stanowisko grafika komputerowego. Odpowiedzialne one będą za stworzenie interfejsu użytkownika. \\
\begin{equation}2 \text{[osoby]} \times 2000 \text{[zł/miesiąc]} \times 4 \text{[miesiąc]} = 16\\\ 000 \text{[zł]} \end{equation} \\
\newpage
\textit{Tabela 3.} Koszty etapu implementacji

\begin{center}
	\begin{tabular}{| l | l |}
		\hline
		\multicolumn{1}{|c|}{Przedsięwzięcie} & 
		\multicolumn{1}{|c|}{Przewidywany koszt} \\ \hline \hline
		Honoraria dla programistów & 144 tys. zł \\ \hline
		Honoraria dla grafików komputerowych & 16 tys. zł \\ \hline
	\end{tabular}
\end{center}

\subsection{Etap testowania}
\noindent
Podczas tego etapu każdy moduł programu będzie testowany przez niezależną grupę testerów oprogramowania. \\
Przewidywana grupa testerów będzie składać się z trzech osób, stawka za miesiąc na osobę wynosi 4000 zł. \\
\begin{equation}3 \text{[osoby]} \times 4000 \text{[zł/miesiąc]} \times 1.5 \text{[miesiąca]} = 18\\\ 000 \text{[zł]} \end{equation} \\
W razie wykrytych błędów za naprawę odpowiadać będzie jedna grupa programistów złożona z trzech osób, stawka za miesiąc na osobę wynosi 4000 zł. \\
\begin{equation}3 \text{[osoby]} \times 4000 \text{[zł/miesiąc]} \times 1.5 \text{[miesiąca]} = 18\\\ 000 \text{[zł]} \end{equation} \\
\textit{Tabela 4.} Koszty etapu testowania

\begin{center}
	\begin{tabular}{| l | l |}
		\hline
		\multicolumn{1}{|c|}{Przedsięwzięcie} & 
		\multicolumn{1}{|c|}{Przewidywany koszt} \\ \hline \hline
		Honoraria dla testerów & 18 tys. zł \\ \hline
		Honoraria dla programistów & 18 tys. zł \\ \hline
	\end{tabular}
\end{center}

\subsection{Podsumowanie}
\noindent
Łączny koszt projektu wynosi \mbox{310\\\ 500 zł.} \\


\newpage
\addcontentsline{toc}{section}{Literatura}
\begin{thebibliography}{2}
	\bibitem{Harmonogram} Kawa M., Przybecki M., Jenda M.: \emph{Dokumentacja witryny internetowej sklepu NUMIZMATYKA. Harmonogram}. Wrocław, Studencka Pracownia Inżynierii Oprogramowania, Instytut Informatyki Uniwersytetu Wrocławskiego~2013.
\end{thebibliography}
\end{document}