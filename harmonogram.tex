%%%%%%%%%%%%%%%%%%%%%%%%%%%%%%%%%%%%%%%%%%%%%%%%%%%%%%%%%%%%%%%%%%%%%%%%%%%%%%%%%%%%%%%%%%
%%%%%%%%%%%%%%%                   PREAMBUŁA / USTAWIENIA               %%%%%%%%%%%%%%%%%%%

\documentclass [11pt, a4paper, leqno]	{article}	% czcionka 11pt, a4, numery formuł do lewej, artykuł
\usepackage [polish]	{babel} 					% ustawiamy jezyk polski dokumentu wynikowego
\usepackage 			{polski}					% zapewniamy sobie poprawne lamanie polski wyrazow
\usepackage [utf8]		{inputenc}					% kodowanie pliku zrodlowego
\usepackage [T1]		{fontenc}					% kodowanie fontow
\usepackage 			{indentfirst}				% wciecie do kazdego akapitu
\usepackage				{setspace}					% interlinia: 1.15
\setstretch {1.15}
\frenchspacing										% przestrzen po zakonczeniu zdania
\renewcommand {\thesection} 		{\arabic{section}.}       	% chcemy kropki na koncu numeracji sekcji
\renewcommand {\thesubsection} 		{\arabic{section}. \arabic{subsection}.}
\renewcommand {\thesubsubsection} 	{\arabic{section}. \arabic{subsection}. \arabic{subsubsection}.}
\textheight 620pt
\makeatletter
\renewcommand\@biblabel[1]{#1.}
\makeatother

%%%%%%%%%%%%%%%%%%%%%%%%%%%%%%%%%%%%%%%%%%%%%%%%%%%%%%%%%%%%%%%%%%%%%%%%%%%%%%%%%%%%%%%%%

\begin{document}

% STRONA TYTUŁOWA

\begin{center}
	\thispagestyle{empty} 							% brak numeracji, naglowka i stopki
	{\large Studencka Pracownia Inżynierii Oprogramowania} 		\\ [0.5cm]
	{\large Instytut Informatyki Uniwersytetu Wrocławskiego} 	\\ [6.0cm]

	{\large Mateusz Kawa, Maciej Przybecki, Marek Jenda} 		\\ [1.5cm]

	{\huge Dokumentacja witryny internetowej sklepu} 			\\ [0.5cm]
	{\huge NUMIZMATYKA} 										\\ [1.5cm]

	{\large Harmonogram} 										\\ [0.5cm]

	\vfill
	
	{\large Wrocław, \today}									\\ [0.5cm]
	{\large Wersja 0.2}
\end{center}

\newpage

% TABELKA WERSJI

\textit{Tabela 1.} Historia zmian dokonanych w dokumencie

\begin{center}
	\begin{tabular}{| l | l | l | l |}
		\hline
		\multicolumn{1}{|c|}{Data} & 
		\multicolumn{1}{|c|}{Numer wersji} &  
		\multicolumn{1}{|c|}{Opis} &
		\multicolumn{1}{|c|}{Autor} \\ \hline \hline
		2013-11-23 & 0.1 & Utworzenie dokumentu & Marek Jenda \\ \hline
		2013-12-03 & 0.2 & Korekta dokumentu  & Maciej Przybecki \\ \hline
	\end{tabular}
\end{center}

\newpage

% SPIS TRESCI

\tableofcontents

\newpage

\section{Wprowadzenie}

\subsection{Cel dokumentu}
\noindent
Niniejszy dokument zawiera ustalenia dotyczące harmonogramu związanego z realizacją projektu witryny internetowej ,,NUMIZMATYKA''.

\section{Harmonogram}
\noindent
Realizacja projektu zgodnie z dobrą praktyką inżynierii oprogramowania została podzielona na kilka etapów. Ze względu na rozmiar przedsięwzięcia oraz dobrą dokumentację pokrewnych projektów przyjęty został model kaskadowy. Prace postępują w sposób sekwencyjny, a każdy etap ma ustalony czas rozpoczęcia i czas zakończenia, których należy przestrzegać.

Projekt został podzielony na następujące etapy:
\begin{itemize}
	\item analiza,
	\item projektowanie,
	\item implementacja,
	\item testowanie,
	\item ocena i przekazanie.
\end{itemize}

Osoby zaangażowane w realizację projektu pracują w dni robocze po 8 godzin zegarowych. Odstęp między każdym z etapów wynosi 2 dni, co chroni przed opóźnieniami z uruchomieniem kolejnego etapu.

\subsection{Etap analizy}
\noindent
Podczas etapu analizy ustalane są wymagania projektu, potrzeby wszystkich grup użytkowników oraz precyzowane są szczegóły dotyczące architektury systemu końcowego. Dokonywane są decyzje dotyczące narzędzi, za pomocą których projektowany będzie system. 

Przewidywany czas potrzebny na realizację fazy analizy wynosi 7 dni. \\


\textit{Tabela 2.} Ramy czasowe etapu analizy

\begin{center}
	\begin{tabular}{| l | l | l |}
		\hline
		\multicolumn{1}{|c|}{Treść} & 
		\multicolumn{1}{|c|}{Data rozpoczęcia} & 
		\multicolumn{1}{|c|}{Data zakończenia} \\ \hline \hline
		Analiza & 2013-10-06 & 2013-10-13 \\ \hline
	\end{tabular}
\end{center}

\subsection{Etap projektowania}
\noindent
W etapie tym definiowana jest architektura sprzętowa i programowa, specyfikowane są parametry związane z wydajnością i bezpieczeństwem systemu. Projektowana jest baza danych oraz wybierane jest środowisko, w którym będzie powstawał system. W tym etapie jest projektowany wstępny interfejs użytkownika. 

Dokonywany jest szczegółowy podział systemu na mniejsze moduły, określane są związki między nimi a także tworzony jest diagram klas w obrębie każdego modułu.

Rezultatem etapu projektowania są specyfikacje, na podstawie których będzie dokonywana implementacja systemu. 

Przewidziany czas trwania etapu projektowania wynosi 30 dni. \\

\textit{Tabela 3.} Ramy czasowe etapu projektowania

\begin{center}
	\begin{tabular}{| l | l | l |}
		\hline
		\multicolumn{1}{|c|}{Treść} & 
		\multicolumn{1}{|c|}{Data rozpoczęcia} & 
		\multicolumn{1}{|c|}{Data zakończenia} \\ \hline \hline
		Projektowanie & 2013-10-16 & 2013-11-15 \\ \hline
	\end{tabular}
\end{center}

\subsection{Etap implementacji}
\noindent
Ten etap obejmuje właściwe konstruowanie systemu na podstawie specyfikacji określonych w poprzednich etapach. Każdy moduł programu jest implementowany przez oddzielną grupę programistów, a także twórców interfejsu użytkownika. 

Rezultatem tego etapu są w pełni działające fragmenty systemu, które po scaleniu dają pełny system.

Przewidziany czas trwania tego etapu wynosi 120 dni. \\

\textit{Tabela 4.} Ramy czasowe etapu implementacji

\begin{center}
	\begin{tabular}{| l | l | l |}
		\hline
		\multicolumn{1}{|c|}{Treść} & 
		\multicolumn{1}{|c|}{Data rozpoczęcia} & 
		\multicolumn{1}{|c|}{Data zakończenia} \\ \hline \hline
		Implementacja & 2013-11-18 & 2014-03-18 \\ \hline
	\end{tabular}
\end{center}

\subsection{Etap testowania}
\noindent
W etapie tym pojedyncze moduły, jak i zintegrowany system, są poddawane testom. Niezależna grupa osób testujących tworzy sytuacje testowe, których celem jest metodycznie weryfikowanie poprawności działania systemu i zgodności ze specyfikacją.

W etapie testowania przewidziana jest możliwość wprowadzania poprawek do systemu w razie wykrytych błędów. 

Czas trwania etapu wynosi 45 dni. \\

\textit{Tabela 5.} Ramy czasowe etapu testowania

\begin{center}
	\begin{tabular}{| l | l | l |}
		\hline
		\multicolumn{1}{|c|}{Treść} & 
		\multicolumn{1}{|c|}{Data rozpoczęcia} & 
		\multicolumn{1}{|c|}{Data zakończenia} \\ \hline \hline
		Testowanie & 2013-03-20 & 2014-05-02 \\ \hline
	\end{tabular}
\end{center}

\subsection{Etap oceny i przekazania}
\noindent
W etapie tym produkt jest prezentowany zleceniodawcy w celu weryfikacji wstępnych założeń. 

\newpage

\addcontentsline{toc}{section}{Literatura}
\begin{thebibliography}{2}
	\bibitem{KrollKruchten07} Kroll P., Kruchten P.: \emph{Rational Unified Process od strony praktycznej}. Warszawa, Wydawnictwa Naukowo--Techniczne~2007.
\end{thebibliography}

\end{document}
