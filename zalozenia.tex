% PREAMBUŁA / USTAWIENIA
\documentclass[11pt,a4paper,leqno]{article}			% czcionka 11pt, a4, wzory do lewej, artykuł
\usepackage[polish]{babel} 				% ustawiamy polskie czcionki
\usepackage[utf8]{inputenc}				% .. i kodowanie
\usepackage[T1]{fontenc}				% kodowanie fontow
\usepackage{setspace}					% insterlinia: 1.5
\frenchspacing						% przestrzen po zakonczeniu zdania
\usepackage{indentfirst}					% wciecie do kazdego akapitu
\renewcommand{\thesection}{\arabic{section}.}       % definiujemy wygodne polecenia
\renewcommand{\thesubsection}{\arabic{section}.\arabic{subsection}.}
\renewcommand{\thesubsubsection}{\arabic{section}.\arabic{subsection}.\arabic{subsubsection}.}

\begin{document}

% STRONA TYTUŁOWA
\begin{center}
\thispagestyle{empty} 					%brak nuemracji, naglowka i stopki
{\Large Studencka Pracownia Inżynierii Oprogramowania}\\[0.5cm]
{\large Instytut Informatyki Uniwersytetu Wrocławskiego}\\[6.0cm]


{\large Mateusz Kawa, Maciej Przybecki, Marek Jenda}\\[1.5cm]
{\huge Dokumentacja witryny internetowej sklepu }\\[0.5cm]
{\huge NUMIZMATYKA }\\[1.5cm]
{\large Założenia ogólne}\\[0.5cm]
\vfill
{\large Wrocław, \today}\\[0.5cm]
{\large Wersja 0.1}
\end{center}

\newpage

% TABELKA WERSJI
\textit{Tabela 1.} Historia zmian dokonanych w dokumencie.
\begin{center}
\begin{tabular}{| l | l | l | l |}
\hline
	\multicolumn{1}{|c|}{Data} & \multicolumn{1}{|c|}{Numer wersji} &  \multicolumn{1}{|c|}{Opis} & \multicolumn{1}{|c|}{Autor} \\ \hline \hline
	2013-10-21 & 0.1 & Utworzenie dokumentu & Marek Jenda \\ \hline
\end{tabular}
\end{center}
\newpage

% SPIS TRESCI
\tableofcontents
\newpage


\newpage
\section{Wprowadzenie}

\subsection{Cel dokumentu}
\noindent
Celem  dokumentu jest prezentacja i opis wymagań ogólnych projektu witryny interentowej sklepu poświęconego tematyce numizmatycznej.
Kwestie dotyczące implemetacji i realizacji konkretnych kroków związanych z budową witryny zostały przedostawione w stosownych dokumentach.

\subsection{Ogólny cel powstania witryny}

\end{document}