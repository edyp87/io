%%%%%%%%%%%%%%%%%%%%%%%%%%%%%%%%%%%%%%%%%%%%%%%%%%%%%%%%%%%%%%%%%%%%%%%%%%%%%%%%%%%%%%%%%%
%%%%%%%%%%%%%%%                   PREAMBUŁA / USTAWIENIA               %%%%%%%%%%%%%%%%%%%

\documentclass [11pt, a4paper, leqno]	{article}	% czcionka 11pt, a4, numery formuł do lewej, artykuł
\usepackage [polish]	{babel} 					% ustawiamy jezyk polski dokumentu wynikowego
\usepackage 			{polski}					% zapewniamy sobie poprawne lamanie polski wyrazow
\usepackage [utf8]		{inputenc}					% kodowanie pliku zrodlowego
\usepackage [T1]		{fontenc}					% kodowanie fontow
\usepackage 			{indentfirst}				% wciecie do kazdego akapitu
\usepackage				{setspace}					% interlinia: 1.15
\setstretch {1.15}
\frenchspacing										% przestrzen po zakonczeniu zdania
\renewcommand {\thesection} 		{\arabic{section}.}       	% chcemy kropki na koncu numeracji sekcji
\renewcommand {\thesubsection} 		{\arabic{section}. \arabic{subsection}.}
\renewcommand {\thesubsubsection} 	{\arabic{section}. \arabic{subsection}. \arabic{subsubsection}.}
\textheight 620pt

%%%%%%%%%%%%%%%%%%%%%%%%%%%%%%%%%%%%%%%%%%%%%%%%%%%%%%%%%%%%%%%%%%%%%%%%%%%%%%%%%%%%%%%%%

\begin{document}

% STRONA TYTUŁOWA

\begin{center}
	\thispagestyle{empty} 							% brak numeracji, naglowka i stopki
	{\large Studencka Pracownia Inżynierii Oprogramowania} 		\\ [0.5cm]
	{\large Instytut Informatyki Uniwersytetu Wrocławskiego} 	\\ [6.0cm]

	{\large Mateusz Kawa, Maciej Przybecki, Marek Jenda} 		\\ [1.5cm]

	{\huge Dokumentacja witryny internetowej sklepu} 			\\ [0.5cm]
	{\huge NUMIZMATYKA} 										\\ [1.5cm]

	{\large Założenia ogólne} 									\\ [0.5cm]

	\vfill
	
	{\large Wrocław, \today}									\\ [0.5cm]
	{\large Wersja 0.2}
\end{center}

\newpage

% TABELKA WERSJI

\textit{Tabela 0.} Historia zmian dokonanych w dokumencie

\begin{center}
	\begin{tabular}{| l | l | l | l |}
		\hline
		\multicolumn{1}{|c|}{Data} & 
		\multicolumn{1}{|c|}{Numer wersji} &  
		\multicolumn{1}{|c|}{Opis} &
		\multicolumn{1}{|c|}{Autor} \\ \hline \hline
		2013-10-21 & 0.1 & Utworzenie dokumentu & Marek Jenda \\ \hline
		2013-11-02 & 0.2 & Korekta & Maciej Przybecki \\ \hline
	\end{tabular}
\end{center}

\newpage

% SPIS TRESCI

\tableofcontents

\newpage

\section{Wprowadzenie}

\subsection{Cel dokumentu}
\noindent
Celem  dokumentu jest prezentacja i opis wymagań ogólnych projektu witryny internetowej sklepu poświęconego tematyce numizmatycznej (dalej nazywanej ,,witryną'').
Szczegółowe kwestie dotyczące implementacji i realizacji konkretnych kroków związanych z budową witryny zostały przedstawione w stosownych dokumentach.

Niniejszy dokument jest przeznaczony dla zleceniodawcy i osób wdrażających witrynę.

\subsection{Ogólny cel powstania witryny}
\noindent
Głównym celem witryny jest umożliwienie wygodnego sposobu dokonywania zakupów przedmiotów związanych z obszarami numizmatyki. Przedmioty umieszczone są w odpowiadających im kategoriach, zatem użytkownik witryny (będący klientem sklepu) może odnaleźć poszukiwany przez niego przedmiot, nawigując ręcznie pośród zgromadzonych w witrynie kategorii; innym sposobem jest skorzystanie z opcji wyszukiwania automatycznego. 

Każdy przedmiot jest umieszczony na przeznaczonej dla niego stronie, na której znajdują się stosowne informacje szczegółowe. 

Użytkownik witryny może dodać wybrany przedmiot do wirtualnej listy zakupów (tzw. koszyk), po czym może kontynuować zakupy lub przejść do procesu finalizowania zakupów. Usługi płatnicze są zarządzane za pośrednictwem zewnętrznych serwisów, np. PayPal. 

Od strony administracyjnej witryna umożliwia dodawanie nowych produktów, a także zarządzanie już istniejącymi.

\section{Opis użytkowników witryny}

\subsection{Kto będzie korzystał z witryny?}
\noindent
Podstawowymi użytkownikami witryny będą klienci sklepu internetowego oraz administratorzy sklepu. 

Klienci sklepu będą mieli możliwość zapoznania się z ofertą oraz dokonania zakupu wybranych przedmiotów. Będą mogli również skontaktować się z obsługą sklepu poprzez wbudowany formularz kontaktowy. Każdy klient będzie mógł założyć konto w sklepie, co umożliwi mu przeglądanie historii transakcji, a także dodawanie istniejących produktów do listy ulubionych.

Administratorzy będą mogli dodawać nowe produkty do sklepu, a także zarządzać już istniejącymi. Ponadto ich zadaniem będzie zarządzanie kontami użytkowników.  


\end{document}
