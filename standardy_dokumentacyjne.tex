%%%%%%%%%%%%%%%%%%%%%%%%%%%%%%%%%%%%%%%%%%%%%%%%%%%%%%%%%%%%%%%%%%%%%%%%%%%%%%%%%%%%%%%%%%
%%%%%%%%%%%%%%%                   PREAMBUŁA / USTAWIENIA               %%%%%%%%%%%%%%%%%%%

\documentclass [11pt, a4paper, leqno]	{article}	% czcionka 11pt, a4, numery formuł do lewej, artykuł
\usepackage [polish]	{babel} 					% ustawiamy jezyk polski dokumentu wynikowego
\usepackage 			{polski}					% zapewniamy sobie poprawne lamanie polski wyrazow
\usepackage [utf8]		{inputenc}					% kodowanie pliku zrodlowego
\usepackage [T1]		{fontenc}					% kodowanie fontow
\usepackage 			{indentfirst}				% wciecie do kazdego akapitu
\usepackage				{setspace}					% interlinia: 1.15
\setstretch {1.15}
\frenchspacing										% przestrzen po zakonczeniu zdania
\renewcommand {\thesection} 		{\arabic{section}.}       	% chcemy kropki na koncu numeracji sekcji
\renewcommand {\thesubsection} 		{\arabic{section}. \arabic{subsection}.}
\renewcommand {\thesubsubsection} 	{\arabic{section}. \arabic{subsection}. \arabic{subsubsection}.}
\textheight 620pt

%%%%%%%%%%%%%%%%%%%%%%%%%%%%%%%%%%%%%%%%%%%%%%%%%%%%%%%%%%%%%%%%%%%%%%%%%%%%%%%%%%%%%%%%%

\begin{document}

% STRONA TYTUŁOWA

\begin{center}
	\thispagestyle{empty} 							% brak numeracji, naglowka i stopki
	{\large Studencka Pracownia Inżynierii Oprogramowania} 		\\ [0.5cm]
	{\large Instytut Informatyki Uniwersytetu Wrocławskiego} 	\\ [6.0cm]

	{\large Mateusz Kawa, Maciej Przybecki, Marek Jenda} 		\\ [1.5cm]

	{\huge Dokumentacja witryny internetowej sklepu} 			\\ [0.5cm]
	{\huge NUMIZMATYKA} 										\\ [1.5cm]

	{\large Standardy dokumentacyjne} 							\\ [0.5cm]

	\vfill
	
	{\large Wrocław, \today}									\\ [0.5cm]
	{\large Wersja 0.1}
\end{center}

\newpage

% TABELKA WERSJI

\textit{Tabela 1.} Historia zmian dokonanych w dokumencie

\begin{center}
	\begin{tabular}{| l | l | l | l |}
		\hline
		\multicolumn{1}{|c|}{Data} & 
		\multicolumn{1}{|c|}{Numer wersji} &  
		\multicolumn{1}{|c|}{Opis} &
		\multicolumn{1}{|c|}{Autor} \\ \hline \hline
		2013-11-24 & 0.1 & Utworzenie dokumentu & Marek Jenda \\ \hline
	\end{tabular}
\end{center}

\newpage

% SPIS TRESCI

\tableofcontents

\newpage

\section{Wprowadzenie}

\subsection{Cel dokumentu}
\noindent
Niniejszy dokument ma na celu ustalenie standardów dokumentacyjnych, wedle których zorganizowana jest dokumentacja realizowanego przedsięwzięcia.
\section{Ogólna struktura dokumentu}
\noindent
Każdy dokument składa się z szeregu ustalonych części w następującej kolejności:
\begin{itemize}
	\item strona tytułowa,
	\item historia zmian dokonanych w dokumencie,
	\item spis treści,
	\item rozdział nakreślający zawartość dokumentu oraz szczegółowe rozdziały dotyczące zagadnień poruszanych w danym dokumencie,
	\item opcjonalnie literatura.
\end{itemize}

Każda z wyżej wymienionych części rozpoczyna się na nowej stronie dokumentu.

\section{Szczegółowa struktura dokumentu}
\subsection{Strona tytułowa}
\noindent
Strona tytułowa została podzielona na trzy fragmenty.

W celu poinformowania odbiorcy o pochodzeniu dokumentu u góry strony tytułowej został umieszczony napis ,,Studencka Pracownia Inżynierii Oprogramowania'', pod którym bezpośrednio identyfikowana jest organizacja w obrębie której dokument został utworzony -- ,,Instytut Informatyki Uniwersytetu Wrocławskiego''.

Na środku strony tytułowej identyfikowani są autorzy projektu, poniżej tytuł całej dokumentacji - ,,Dokumentacja witryny internetowej sklepu NUMIZMATYKA''.
Następnie określony jest temat konkretnego dokumentu. 

U dołu strony podane jest miasto w którym dokumentacja została utworzona, data ostatniej modyfikacji, a poniżej numer wersji dokumentu.

\subsection{Historia zmian dokonanych w dokumencie}
\noindent
W części tej znajduje się tabela przedstawiająca historię zmian konkretnego dokumentu. Nad tabelą znajduje się jej numer oraz opis. Sama tabela składa się z czterech kolumn określonych w pierwszym wierszu tabeli:
\begin{itemize}
	\item ,,Data'' -- data w standardzie \textit{ISO 8601} czyli YYYY-MM-DD. Data określa dzień, w którym dokonana została zmiana,
	\item ,,Numer wersji'' -- numer wersji po konkretnej zmianie,
	\item ,,Opis'' -- krótki opis dokonanej zmiany,
	\item ,,Autor'' -- imię i nazwisko osoby, która dokonała zmian w dokumencie. Jeżeli osób było więcej, imiona i nazwiska oddzielone są przecinkiem.
\end{itemize}

Pierwszym numerem wersji zawsze jest 0.1, a zmiana ta opisana jest jako ,,Utworzenie dokumentu''. Numery wersji i daty uporządkowane są rosnąco, a numer wersji z ostatniego wiersza tabeli widnieje również na stronie tytułowej dokumentu.

\subsection{Spis treści}
\noindent
Na górze strony ze spisem treści znajduje się napis ,,Spis treści'', natomiast poniżej znajduje się właściwy spis rozdziałów znajdujących się w dokumencie. 

\subsection{Rozdział nakreślający treść dokumentu}
\noindent
W rozdziale tym opisana jest zwięźle zawartość dokumentu wprowadzająca czytelnika w tematykę poruszaną w dalszych rozdziałach.

\subsection{Pozostałe rozdziały}
\noindent
W pozostałych rozdziałach szczegółowo omawiane są kwestie dotyczące tematyki poruszanej w dokumencie.

\subsection{Sekcja bibliograficzna}
\noindent
Sekcja ta rozpoczyna się napisem ,,Literatura'', pod którą, w liście numerowanej, umieszczone są pojedyncze wpisy z notami bibliograficznymi.
Każdy wpis rozpoczyna się numerem umieszczonym w nawiasie kwadratowym. Właściwy wpis formatowany jest zgodnie z normą ISO 690, która opisana została we odpowiedniej dokumentacji. 

\section{Szczegółowe wytyczne dotyczące rozdziałów}

\subsection{Czcionka i dostępy}
\noindent
Domyślnym rozmiarem czcionki stosowanym w dokumencie jest 11 punktów typograficznych (dalej - pt). Identyfikatory rozdziałów i podrozdziałów są pogrubione, ponadto tytuły rozdziałów głównych pisane są czcionką 15pt, natomiast podrozdziałów - 13pt.

Domyślnym odstępem stosowanym w dokumencie jest światło międzywierszowe o wielkości 1,15. Odstęp między tytułem rozdziału a jego zawartością wynosi 1,5 natomiast odstęp między rozdziałem poprzednim a kolejnym tytułem -- 2.
	
Każdy paragraf w rozdziałach rozpoczyna się wcięciem. Wyjątkiem jest pierwszy paragraf, który w całości wyrównany jest do lewego marginesu.  Tekst w obrębie rozdziału jest wyrównany do lewego i prawego marginesu.

\subsection{Numeracja rozdziałów}
\noindent
Rozdziały w całym dokumencie numerowane są cyframi arabskimi a każdy rozdział niższego poziomu identyfikowany jest kolejną liczbą arabską oddzieloną od poprzedniej liczby pojedynczą kropką i odstępem. 

Jeżeli w dokumencie występuje sekcja bibliograficzna, to jest ona opisana jako ,,Literatura'' i traktowana jest jako rozdział najwyższego poziomu bez określonej numeracji. 

\subsection{Tabele}
\noindent
Nad każdą tabelą znajduje się jej etykieta składająca się z pochyłego napisu ,,Tabela XX. '' gdzie XX jest kolejnym numerem tabeli w tekście, po którym następuje krótki opis tabeli. Opis nie jest zakończony kropką. 

W tabeli znajduje się tylko tekst, w pierwszym wierszu umieszczone są wyśrodkowane etykiety kolumn, natomiast w kolejnych wierszach umieszczona jest właściwa zawartość tabeli. 

\newpage
\addcontentsline{toc}{section}{Literatura}
\begin{thebibliography}{2}
	\bibitem{KrollKruchten07} Kroll P., Kruchten P., \emph{Rational Unified Process od strony praktycznej}, Warszawa, Wydawnictwo naukowo techniczne~2007.
	\bibitem{NeckaStocki09} Nęcka E., Stocki R., \emph{Jak pisać prace z psychologii - poradnik dla studentów i badaczy}, Kraków, Universitas~2009.
	\bibitem{IEEE830} IEEE Std 830-1998. \emph{IEEE Recommended Practice for Software Requirements Specifications}.
	\bibitem{PNISO690} PN-ISO 690:2002. \emph{Dokumentacja – Przypisy bibliograficzne – Zawartość, forma i struktura}.
	\bibitem{ISO8601} ISO 8601. \emph{Date and Time Formats}.
\end{thebibliography}

\end{document}
