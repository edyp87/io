%%%%%%%%%%%%%%%%%%%%%%%%%%%%%%%%%%%%%%%%%%%%%%%%%%%%%%%%%%%%%%%%%%%%%%%%%%%%%%%%%%%%%%%%%%
%%%%%%%%%%%%%%%                   PREAMBUŁA / USTAWIENIA               %%%%%%%%%%%%%%%%%%%

\documentclass [11pt, a4paper, leqno]	{article}	% czcionka 11pt, a4, numery formuł do lewej, artykuł
\usepackage [polish]	{babel} 					% ustawiamy jezyk polski dokumentu wynikowego
\usepackage 			{polski}					% zapewniamy sobie poprawne lamanie polski wyrazow
\usepackage [utf8]		{inputenc}					% kodowanie pliku zrodlowego
\usepackage [T1]		{fontenc}					% kodowanie fontow
\usepackage 			{indentfirst}				% wciecie do kazdego akapitu
\usepackage				{setspace}					% interlinia: 1.15
\setstretch {1.15}
\frenchspacing										% przestrzen po zakonczeniu zdania
\renewcommand {\thesection} 		{\arabic{section}.}       	% chcemy kropki na koncu numeracji sekcji
\renewcommand {\thesubsection} 		{\arabic{section}. \arabic{subsection}.}
\renewcommand {\thesubsubsection} 	{\arabic{section}. \arabic{subsection}. \arabic{subsubsection}.}
\textheight 620pt

%%%%%%%%%%%%%%%%%%%%%%%%%%%%%%%%%%%%%%%%%%%%%%%%%%%%%%%%%%%%%%%%%%%%%%%%%%%%%%%%%%%%%%%%%

\begin{document}

% STRONA TYTUŁOWA

\begin{center}
	\thispagestyle{empty} 							% brak numeracji, naglowka i stopki
	{\large Studencka Pracownia Inżynierii Oprogramowania} 		\\ [0.5cm]
	{\large Instytut Informatyki Uniwersytetu Wrocławskiego} 	\\ [6.0cm]

	{\large Mateusz Kawa, Maciej Przybecki, Marek Jenda} 		\\ [1.5cm]

	{\huge Dokumentacja witryny internetowej sklepu} 			\\ [0.5cm]
	{\huge NUMIZMATYKA} 										\\ [1.5cm]

	{\large Kosztorys} 										\\ [0.5cm]

	\vfill
	
	{\large Wrocław, \today}									\\ [0.5cm]
	{\large Wersja 0.1}
\end{center}

\newpage

% TABELKA WERSJI

\textit{Tabela 0.} Historia zmian dokonanych w dokumencie

\begin{center}
	\begin{tabular}{| l | l | l | l |}
		\hline
		\multicolumn{1}{|c|}{Data} & 
		\multicolumn{1}{|c|}{Numer wersji} &  
		\multicolumn{1}{|c|}{Opis} &
		\multicolumn{1}{|c|}{Autor} \\ \hline \hline
		2013-11-26 & 0.1 & Utworzenie dokumentu & Maciej Przybecki \\ \hline
	\end{tabular}
\end{center}
\newpage

% SPIS TRESCI

\tableofcontents

\newpage

\section{Wprowadzenie}

\subsection{Cel dokumentu}
\noindent
Niniejszy dokument zawiera ustalony kosztorys z realizacją projektu witryny internetowej ,,NUMIZMATYKA''. \\

\section{Kosztorys}
\noindent
Realizacja projektu zgodnie z dobrą praktyką inżynierii oprogramowania została podzielona na kilka etapów. Ze względu na rozmiar przedsięwzięcia oraz dobrą dokumentację pokrewnych projektów przyjęty został model kaskadowy. Prace postępują w sposób sekwencyjny, a każdy etap ma ustalony czas rozpoczęcia i czas zakończenia, których należy przestrzegać. \\

Projekt został podzielony na następujące etapy:
\begin{itemize}
\item Analiza,
\item Projektowanie,
\item Implementacja,
\item Testowanie,
\item Ewaluacja i przekazanie.
\end{itemize}


\subsection{Etap analizy}
\noindent
Na tym etapie wymagane jest przeprowadzenie badania rynku, oraz zakup odpowiednich narzędzi przy pomocy, których projektowany będzie system. \\
\textit{Tabela 1.} Koszty dla etapu analizy

\begin{center}
	\begin{tabular}{| l | l |}
		\hline
		\multicolumn{1}{|c|}{Przedsięwzięcie} & 
		\multicolumn{1}{|c|}{Przewidywany koszt} \\ \hline \hline
		Przeprowadzenie ankiety elektronicznej na docelowej grupie odbiorczej & 5 tys. PLN \\ \hline
		Zakup niezbędnego oprogramowania & 10 tys. PLN \\ \hline
	\end{tabular}
\end{center}

\subsection{Etap projektowania}
\noindent
Podczas tego etapu wymagane jest zatrudnienie 3 osób odpowiedzialnych za zaprojektowanie całego systemu. \\
\begin{equation}3 [osoby] \times  8'500 [PLN/miesiąc] \times 1 [miesiąc] = 25'500 [PLN] \end{equation} \\
Wykonany jest także zakup odpowiednichnarzędzi przy pomocy, których implementowany będzie system.  \\
Zatrudnione będą także 2 osoby na stanowisko grafika komputerowego. Odpowiedzialne one będą za zaprojektowanie interfejsu użytkownika. \\
\begin{equation}2 [osoby] \times 2'000 [PLN/miesiąc] \times 1 [miesiąc] = 4'000 [PLN] \end{equation} \\
\textit{Tabela 2.} Koszty dla etapu projektowania

\begin{center}
	\begin{tabular}{| l | l |}
		\hline
		\multicolumn{1}{|c|}{Przedsięwzięcie} & 
		\multicolumn{1}{|c|}{Przewidywany koszt} \\ \hline \hline
		Zakup serwera & 50 tys. PLN \\ \hline
		Pieniądze przeznaczone na wypłaty dla projektantów & 25.5 tys. PLN \\ \hline
		Zakup niezbędnego oprogramowania & 20 tys. PLN \\ \hline
		Pieniądze przeznaczone na wypłaty dla grafików komputerowych & 4 tys. PLN \\ \hline
	\end{tabular}
\end{center}

\subsection{Etap implementacji}
\noindent
Podczas tego etapu każdy moduł programu będzie implementowany przez oddzielną grupę programistów. \\
Przewidywane są 3 grupy programistów składające się z 3 osób, stawka za miesiąc na osobę wynosi 4’000 PLN. \\
\begin{equation}3 [osoby] \times 3 osoby \times 4'000 [PLN/miesiąc] \times  4 [miesiąc] = 144'000 [PLN] \end{equation} \\
Zatrudnione będą także 2 osoby na stanowisko grafika komputerowego. Odpowiedzialne one będą za stworzenie interfejsu użytkownika. \\
\begin{equation}2 [osoby] \times 2'000 [PLN/miesiąc] \times 4 [miesiąc] = 16'000 [PLN] \end{equation} \\
\textit{Tabela 3.} Koszty dla etapu implementacji

\begin{center}
	\begin{tabular}{| l | l |}
		\hline
		\multicolumn{1}{|c|}{Przedsięwzięcie} & 
		\multicolumn{1}{|c|}{Przewidywany koszt} \\ \hline \hline
		Pieniądze przeznaczone na wypłaty dla programistów & 144 tys. PLN \\ \hline
		Pieniądze przeznaczone na wypłaty dla grafików komputerowych & 16 tys. PLN \\ \hline
	\end{tabular}
\end{center}

\subsection{Etap testowania}
\noindent
Podczas tego etapu każdy moduł programu będzie testowany przez niezależną grupę testerów oprogramowania. \\
Przewidywana grupa testerów będzie składać się z 3 osób, stawka za miesiąc na osobę wynosi 4’000 PLN. \\
\begin{equation}3 [osoby] \times 4'000 [PLN/miesiąc] \times 1.5 [miesiąca] = 18'000 [PLN] \end{equation} \\
W razie wykrytych błędów, za naprawę odpowiadać będzie jedna grupa programistów złożona z 3 osób, stawka za miesiąc na osobę wynosi 4’000 PLN. \\
\begin{equation}3 [osoby] \times 4'000 [PLN/miesiąc] \times 1.5 [miesiąca] = 18'000 [PLN] \end{equation} \\
\textit{Tabela 4.} Koszty dla etapu testowania

\begin{center}
	\begin{tabular}{| l | l |}
		\hline
		\multicolumn{1}{|c|}{Przedsięwzięcie} & 
		\multicolumn{1}{|c|}{Przewidywany koszt} \\ \hline \hline
		Pieniądze przeznaczone na wypłaty dla testerów & 18 tys. PLN \\ \hline
		Pieniądze przeznaczone na wypłaty dla programistów & 18 tys. PLN \\ \hline
	\end{tabular}
\end{center}

\subsection{Podsumowanie}
\noindent
Sumaryczny koszt projektu wynosi: 310'500 PLN. \\

\addcontentsline{toc}{section}{Literatura}
\begin{thebibliography}{2}
	\bibitem{Harmonogram} Kawa M., Przybecki M., Jenda M., \emph{Dokumentacja witryny internetowej sklepu NUMIZMATYKA Harmonogram}, Wrocław, Studencka Pracownia Inżynierii Oprogramowania Instytut Informatyki Uniwersytetu Wrocławskiego~2013.
\end{thebibliography}
\end{document}